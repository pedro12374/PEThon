
\section{Variáveis}
Quando começamos a mexer com programação umas das mais fundamentais e primordiais ferramentas que tem-se são as variáveis, estás que estão presentes em todas as linguagens de programação, e praticamente todos os códigos já feitos.
\subsection{O que são variáveis?}
Como o próprio nome já diz, são coisa que variam ou podem variar. Basicamente são pedaços da memoria reservados para receberem algo, um numero inteiro, real, uma letra, uma frase, uma conta, qualquer coisa pode ser atribuída a uma variável, e logo você vai entender como utiliza-las para poder obter um melhor desempenho na hora de resolver algum problema.
\subsection{Tipos de Variáveis}
Em python temos alguns tipos de variáveis, ou tipos de dados estes são: 
\begin{itemize}
	\item str(string): É um ou mais caracteres demarcados por aspas, podem ser letras, palavras, frases, qualquer texto.
	
	Ex: 'Hello World!', 'a','Chucrute'

	\item int(inteiro): São números inteiros
	
	Ex: 1, 2, 154654, 3, 0
	\item float(ponto flutuante): Números decimais ou inteiros com zero, um detalhe é que em python, e todas as outras linguagens de programação, é utilizado o ponto para separar números decimais e não a virgula
	
	Ex: 3.141592, 96.654, 1.0, 2.3
	\item list: É o chamado de agregador homógeno, uma lista reúne vários itens em uma só variável, como uma lista de str, int ou mesmo uma lista de listas ou todos esses tipos juntos.

	Ex: ['Abacaxi','Uva','Maça'], ['9.8','7.6','3.2'], 	[4.0, 'Jacaré', True]

	\item tupla: A tupla é um tipo de lista, só que imutável, as str são um tipo especial de tupla, pois ela agrega um mesmo tipo de caracteres juntos, e não pode ser modificada.

	Ex: ('Rosa','Verde','Azul')

	\item dict(Dicionário): É um tipo de lista, onde ao invés de se ter acesso ao itens através da posição dos mesmos, se utiliza uma chave para acessar o item.

	Ex:	\{'Olho': Azul, 'Altura': 1.70, 'Cabelo':Castanho\}

	\item bool(lógico): Tipo de lógico de dados

	Ex: True, False
\end{itemize}

\subsection{Utilizando uma variável}
Agora que você já sabe quais são os tipos de dados e para que usamos cada um deles, vamos aprender a usa-los
\subsubsection{Criando uma variável}
Antes de sair utilizando uma variável você deve primeiro cria-la e então por ela para o serviço, para criar uma variável primeiro deve-se dar um nome para ela, porem para isso há algumas regras a se seguir, que são:
\begin{itemize}
\item Não pode se começar um nome de variável com:

	\begin{itemize}
	\item Números
	\item Caracteres com acentos
	\item Espaços
	\item Caracteres especiais($\$,\#,\%,@,\!,\&,(,),\{,\},[,],...$)
	\end{itemize}

\item Recomendações:
	\begin{itemize}
	\item Começar com letras maiúsculas
	\item Nomes descritivos
	\end{itemize}
\end{itemize}

Agora que você já sabe como nomear a sua variável, está na hora de cria-la definitivamente. Diferente de outras linguagens de programação, em Python, você não necessita especificar que está criando uma variável, ou o tipo da mesmo, só é necessário dar um nome e atribuir a ela um valor. Crie duas variáveis com os nomes que preferir e de um valor para cada.
\code{a = 10

b = 15}

Criada as variáveis, está na hora de utiliza-las e entender como funciona as coisas. Vamos exibir a primeira variável;
\code{$print$ a}

Agora execute o seu programa e veja o resultado.

\subsubsection{Utilizando variáveis}
 
 Agora que você já sabe criar as suas próprias variáveis, está na hora de aprender a manipula-las a para seu beneficio. As variáveis tem algumas propriedades que são muito uteis e as definem, algumas delas são:


\begin{itemize}
	\item Seu valor pode ser alterado, mas seu tipo não;
	\item Operações entre variáveis;
	\item Operações com números;
	\item Tipos de variáveis diferentes podem realizar operações juntas;
	\item Entre outros ...
\end{itemize}

Além de definir o valor direto de uma variável, é possível fazer isso usando outras variáveis, assim expandindo as possibilidades, por exemplo:

\code{
	a = 5

	b = 9

	c = a + b

	print a,b,c
}

Execute o comando a cima e veja o que ocorre. Você acabou de definir duas variáveis \itc{a} e \itc{b} utilizando valores próprios, e então passou o valor da soma das duas para a variável \itc{c} sem a necessidade de utilizar os valores das variáveis já declaradas, e caso o valor de cada uma delas se alterasse, o valor de \itc{c} se ajustaria, pois ela esta vinculada a soma de \itc{a} e \itc{b}.